\documentclass[a4paper]{article}
\usepackage[italian]{babel}
\usepackage[utf8]{inputenc}
\usepackage[margin=1cm, top=0cm]{geometry}
\usepackage{amsmath}

\title{\textbf{Documentazione elaborato\\Reti di Telecomunicazione 2024}}
\author{Luca Casadei - 0001069237\\Francesco Pazzaglia - 0001077423}
\date{Ultima modifica: \today}
\pagenumbering{gobble}

\begin{document}
	\maketitle
	\section{Descrizione Progetto}
	Si vuole realizzare un'infrastruttura di rete composta da diversi settori, in particolare, 2 interni, un collegamento esterno con una rete Internet simulata ed una rete wireless a cui si può accedere pubblicamente. Si suppone che a ciascuna rete interna non siano collegati più di 50 apparati, mentre la rete pubblica ne supporta molti di più. Nel complesso, l'organizzazione delle reti può essere riassunta nella seguente maniera:
	\begin{itemize}
		\item Area \textbf{Ufficio}: Suddivisa attraverso VLAN (3), vengono simulati 2 apparati client (PC e stampante) e un apparato Server, quest'ultimo mette a disposizione una pagina web da visualizzare (Web-Server).
		\item Area \textbf{Ospiti}: Suddivisa attraverso VLAN (2), vengono simulati 5 apparati, di cui 3 con collegamento via cavo e 2 collegati via etere.
		\item Area \textbf{WiFiPubblico}: Su router separato, vengono simulati 4 apparati tutti collegati attraverso etere ad un access point, su questa rete non viene configurata alcuna VLAN.
	\end{itemize}
	Dato lo scopo didattico dell'esercitazione, quindi per verificare la corretta configurazione dei protocolli di routing dei gateway e delle VLAN nello switch, tutti i dispositivi vengono fatti comunicare tra loro, in una rete realistica, questo non avviene.
	\section{Indirizzamento}
	Dato il numero di apparati gestibili dalle reti interne, come subnet mask sono stati scelti 26 bit per le reti, e 6 bit per gli host, in modo tale da poter gestire $(2^{6} - 2)$ macchine e $2^{26}$ reti. La maschera scelta è quindi $255.255.255.192$ e la relativa wildcard $0.0.0.63$ usata per il protocollo di routing \textit{OSPF} configurato nei due gateway.\\
	Dovendo gestire molti più host, per la rete pubblica è stato usato l'indirizzo di classe A con maschera $255.0.0.0$ e wildcard $0.255.255.255$ per OSPF.\\
	Indirizzamento dettagliato delle reti:
	\begin{itemize}
		\item Area \textbf{Ufficio}:
		\begin{itemize}
			\item Rete: 192.168.1.192/26
			\item DG: 192.168.1.254
		\end{itemize}
		\item Area \textbf{Ospiti}:
		\begin{itemize}
			\item Rete: 192.168.1.128/26
			\item DG: 192.168.1.190
		\end{itemize}
		\item Area \textbf{WiFiPubblico}:
		\begin{itemize}
			\item Rete: 10.0.0.0/8
			\item DG: 10.255.255.254
		\end{itemize}
		\item Area tra gateway:
		\begin{itemize}
			\item Rete: 12.0.0.0/8
			\item Router1: 12.255.255.254
			\item Router0: 12.0.0.1
		\end{itemize}
		\item Area \textbf{Internet}:
		\begin{itemize}
			\item Rete: 11.0.0.0/8
			\item Router1: 12.255.255.253
		\end{itemize}
	\end{itemize}
	Ulteriori configurazioni sono visualizzabili dalla lista dei comandi.
\end{document}